% Created 2023-07-16 dom 10:38
% Intended LaTeX compiler: pdflatex
\documentclass[letterpaper, 12pt]{article}
                      \usepackage{lmodern} % Ensures we have the right font
\usepackage[T1]{fontenc}
\usepackage[utf8]{inputenc}
\usepackage{graphicx}
\usepackage{amsmath, amsthm, amssymb}
\usepackage[table, xcdraw]{xcolor}
\usepackage{algorithm}
\usepackage{algpseudocode}
\usepackage{titling}
\setlength{\droptitle}{-6em}
\setlength{\parindent}{0pt}
\setlength{\parskip}{1em}
\usepackage[stretch=10]{microtype}
\usepackage{hyphenat}
\usepackage{ragged2e}
\usepackage{subfig} % Subfigures (not needed in Org I think)
\usepackage{hyperref} % Links
\usepackage{listings} % Code highlighting
\usepackage[top=1in, bottom=1.25in, left=1in, right=1in]{geometry}
\renewcommand{\baselinestretch}{1.3}
\usepackage[explicit]{titlesec}
\pretitle{\begin{center}\fontsize{20pt}{20pt}\selectfont}
\posttitle{\par\end{center}}
\preauthor{\begin{center}\vspace{-6bp}\fontsize{16pt}{16pt}\selectfont}
\postauthor{\par\end{center}\vspace{-25bp}}
\predate{\begin{center}\fontsize{12pt}{12pt}\selectfont}
\postdate{\par\end{center}\vspace{0em}}
\titlespacing\section{0pt}{5pt}{5pt} % left margin, space before section header, space after section header
\titlespacing\subsection{0pt}{5pt}{-2pt} % left margin, space before subsection header, space after subsection header
\titlespacing\subsubsection{0pt}{5pt}{-2pt} % left margin, space before subsection header, space after subsection header
\usepackage{enumitem}
\setlist{itemsep=-2pt} % or \setlist{noitemsep} to leave space around whole list
\titleformat{\section} {\huge}{\thesection}{1em}{\textbf{#1}} % Section header formatting
\titlespacing\section{0pt}{5pt}{-5pt} % left margin, space before section header, space after section header
\titleformat{\subsection} {\large}{\thesubsection}{1em}{\textbf{#1}}
\titleformat{\subsubsection} {\large}{\thesubsubsection}{1em}{#1}
\setcounter{secnumdepth}{2}
\author{Matteo Lugli}
\date{}
\title{Gestione di veicoli autonomi agli incroci}
\hypersetup{
 pdfauthor={Matteo Lugli},
 pdftitle={Gestione di veicoli autonomi agli incroci},
 pdfkeywords={},
 pdfsubject={},
 pdfcreator={Emacs 27.1 (Org mode 9.3)}, 
 pdflang={English}}
\begin{document}

\maketitle
\tableofcontents

\newpage
\section{Introduzione}
\label{sec:orgc7851ce}
La guida autonoma nei prossimi anni diventerà progressivamente parte
integrante della nostra vita, sostituendo una delle attività umane 
tra le più stressanti e pericolose. 
Soltanto nel 2021 in Italia sono stati registrati più di 150 mila incidenti
stradali che hanno provocato circa 3000 morti e 200 mila feriti.
Il ruolo principale della tecnologia è quello di creare soluzioni che mettano
l'esperienza dell'uomo al centro, cercando di facilitare, migliorare e rendere 
più sicure le attività quotidiane.
A tal proposito la ricerca sui veicoli autonomi si è evoluta molto rapidamente negli ultimi anni,
proponendosi di trasformare radicalmente il modo in cui guidiamo.
La Society of Automotive Engineers (SAE) descrive il coinvolgimento umano alla guida tramite
una classifica basata su 5 livelli:
\begin{itemize}
\item Livello 0 \newline
Nessuna automazione: le auto non presentano dispositivi automatici in grado di dare
assistenza attiva al guidatore, che deve sempre mantenere il pieno controllo del veicolo.
Sono ricondotte a questo livello anche automobili equipaggiate con sistemi di frenata di 
emergenza o di rilevazione di collisioni imminenti.
\item Livello 1 \newline
Assistenza alla guida: il veicolo è dotato di sistemi come l'assistenza al mantenimento della
corsia (LKA) o il controllo di velocità adattivo (ACC). Il conducente è comumque pienamente
responsabile della guida, mantenendo il controllo del veicolo.
\item Livello 2 \newline
Automazione parziale alla guida: i veicoli sono equipaggiati con sistemi ADAS (\emph{Advanced Driver Assistance Systems})
che in certe situazioni possono aiutare attivamente il conducente con il controllo dello sterzo, l'accellerazione
e la frenata. \emph{Autopilot} di \emph{Tesla}, che ad oggi viene continuamente aggiornato tramite
l'aggiunta di nuove funzionalità di assistenza alla guida, è ancora classificato come livello 2.
\item Livello 3 \newline
Automazione condizionata: il veicolo è in grado di gestire autonomamente la guida in
strada. Per legge, il conducente deve comunque essere sempre abilitato a poter mettere
le mani sul volante e prendere il controllo dell'auto. Alcune compagnie
come \emph{Honda} hanno rilasciato sul mercato automobili di livello 3 considerabili
completamente autonome in alcune zone urbane del giappone o sulle autostrade.
\item Livello 4 \newline
Automazione elevata: il veicolo è completamente autonomo nella guida e nella navigazione.
Nonostante sia comunque necessaria la presenza di un conducente, il suo intervento
non dovrebbe essere mai richiesto. Anche questi sistemi sono tipicamente funzionanti
soltanto in certe aree geografiche.
\item Livello 5 \newline
Automazione completa: il veicolo è autonomo in qualsiasi condizione, e non è necessaria
la presenza di un passeggero a bordo.
\end{itemize}
\subsection{Contributo di questa tesi}
\label{sec:orgb47c539}
Questa tesi vuole esplorare ulteriori possibilità offerte dagli algoritmi
di coordinamento di veicoli autonomi agli incroci basati sulle aste.
In particolare, vengono proposte alcune estensioni dell'algoritmo
cooperativo spiegato in [paper gherardini] per cercare di migliorare
\emph{(i)} il \emph{throughput} dell'incrocio e \emph{(ii)} il tempo di attesa nel 
traffico dei veicoli tramite una migliore gestione indiretta delle situazioni
di congestione.
Si è poi proposto un algoritmo basato sul \emph{Deep Q-Learning}[paper deepmind]
in grado di allenare i manager degli incroci basandosi soltanto sull'esperienza
passata, senza avere a disposizione dati di partenza.
In particolare, si è voluto verificare che l'overhead dato dalla presenza di 
un complesso meccanismo economico basato sulle aste fosse realmente necessario per mantenere
la \emph{fairness} dell'incrocio.
\newpage
\section{Algoritmo}
\label{sec:org4f8c7a3}
\begin{algorithm}
\caption{Algoritmo ad asta ibrida}\label{alg:cap}
\begin{algorithmic}
\Require $n \geq 0$
\Ensure $y = x^n$
\State $y \gets 1$
\State $X \gets x$
\State $N \gets n$
\While{$N \neq 0$}
\If{$N$ is even}
    \State $X \gets X \times X$
    \State $N \gets \frac{N}{2}$  \Comment{This is a comment}
\ElsIf{$N$ is odd}
    \State $y \gets y \times X$
    \State $N \gets N - 1$
\EndIf
\EndWhile
\end{algorithmic}
\end{algorithm}
\end{document}
